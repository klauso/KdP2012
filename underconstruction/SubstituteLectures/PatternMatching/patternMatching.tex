\documentclass[12pt]{beamer}

\usefonttheme[onlymath]{serif}
% \usetheme{Hannover}
% \usetheme{default}

\setbeamertemplate{frametitle}{
 \begin{centering}
  \smallskip
  \insertframetitle\par
  \smallskip
 \end{centering}
}
\setbeamertemplate{itemize item}{$\bullet$}
\setbeamertemplate{navigation symbols}{}
\setbeamertemplate{footline}[text line]{
 \hfill
 \strut{
  \scriptsize\sf\color{black!60}
  \insertframenumber
 }
 \hfill
}

% Define some colors:
\definecolor{DarkFern}{HTML}{407428}
\definecolor{DarkCharcoal}{HTML}{4D4944}
\colorlet{Fern}{DarkFern!85!white}
\colorlet{Charcoal}{DarkCharcoal!85!white}
\colorlet{LightCharcoal}{Charcoal!50!white}
\colorlet{AlertColor}{orange!80!black}
\colorlet{DarkRed}{red!70!black}
\colorlet{DarkBlue}{blue!70!black}
\colorlet{DarkGreen}{green!70!black}

% Use the colors:
\setbeamercolor{title}{fg=Fern}
\setbeamercolor{frametitle}{fg=Fern}
\setbeamercolor{normal text}{fg=Charcoal}
\setbeamercolor{block title}{fg=black,bg=Fern!25!white}
\setbeamercolor{block body}{fg=black,bg=Fern!25!white}
\setbeamercolor{alerted text}{fg=AlertColor}
\setbeamercolor{itemize item}{fg=Charcoal}

\usepackage[utf8]{inputenc}
\usepackage{kerkis}
\usepackage{mintln}
\usepackage{verbments}
\usepackage{parskip}
% \parskip=0.25\baselineskip
\usepackage{multirow}
\usepackage{color}

\definecolor{strcol}{HTML}{ba2121}

\newcommand{\hl}[1]{\textcolor{black}{#1}}

\title{\bf Pattern Matching}
\author{Yi Dai}
\date{\today}

\begin{document}

\begin{frame}[plain]
 \maketitle
\end{frame}

\begin{frame}[fragile]{{\bf Motivating Example 1 --- Debitalization} (1/2)}
 \large
 \begin{table}
  \centering
  \begin{tabular}{ccc}
   bit & $\overset{debitalize}{\longrightarrow}$ & truth \\
   \hline
   0   &                                         & false \\
   1   &                                         & true  \\
  \end{tabular}
 \end{table}
\end{frame}

\section{Matching Literals}

\begin{frame}[fragile]{{\bf Debitalization} (1a/2)}
 Racketize it

 \pause

 Data
 \begin{itemize}
  \item number: \mint{scheme}|0|, \mint{scheme}|1|
  \item boolean: \mint{scheme}|false|, \mint{scheme}|true|
 \end{itemize}

 \pause

 Code
 {\footnotesize
  \begin{minted}{scheme}
   ;; debit : number -> boolean
   ;; debitalizes a bit, via equality test
   (define (debit b)
     (cond ((= b 0) false)
           ((= b 1) true ) ) )
  \end{minted}
 }
\end{frame}

\begin{frame}[fragile]{{\bf Debitalization} (1b/2)}
 Bit operators
 \begin{table}
  \centering
  \begin{tabular}{|c|c|}
   \hline
   bit operator & boolean operator \\
   \hline
   NOT & $\neg$ \\
   \hline
   AND & $\wedge$ \\
   \hline
   OR  & $\vee$  \\
   \hline
  \end{tabular}
 \end{table}

 \pause
 
 Racket code
 {\footnotesize
  \begin{minted}{scheme}
   ;; NOT : number -> number
   ;; negates a bit, via equality test
   ;; ...
  \end{minted}
 }

 \pause

 \mint{scheme}|AND| and \mint{scheme}|OR| ...
\end{frame}

\begin{frame}[fragile]{{\bf Debitalization} (1c/2)}
 ...

 \pause

 {\footnotesize
  \begin{minted}{scheme}
   ;; AND : number number -> number
   ;; conjoins two bits, via equality test
   (define (AND b1 b2)
     (cond ((and (= b1 0) (= b2 0)) 0)
           ((and (= b1 0) (= b2 1)) 0)
           ((and (= b1 1) (= b2 0)) 0)
           ((and (= b1 1) (= b2 1)) 1) ) )

   ;; OR : number number -> number
   ;; disjoins two bits, via equality test
   ;; ...

   (debit (NOT (OR (AND 0 1) (AND 1 0))))
  \end{minted}
 }
\end{frame}

\begin{frame}[fragile]{\bf A First Taste of Pattern Matching}
 Refactor it
 {\footnotesize
  \begin{minted}{scheme}
   ;; debit : number -> boolean
   ;; debitalizes a bit, via pattern matching
   (define (debit b)
     (match b
       (0 false)
       (1 true ) ) )
  \end{minted}
 }

 \pause

 {\footnotesize
  \begin{minted}{scheme}
   ;; NOT : number -> number
   ;; negates a bit, via pattern matching
   ;; ...
  \end{minted}
 }

 \pause

 \mint{scheme}|AND| and \mint{scheme}|OR| ...
\end{frame}

\begin{frame}[fragile]{\bf Nested Pattern Matching}
 ...

 {\footnotesize
  \begin{minted}{scheme}
   ;; AND : number number -> number
   ;; conjoins two bits, via pattern matching
   (define (AND b1 b2)
     (match b1
       (0 0)
       (1 (match b2
            (0 0)
            (1 1) ) ) ) )

   ;; OR : number number -> number
   ;; disjoins two bits, via pattern matching
   ;; ...

   (debit (NOT (OR (AND 0 1) (AND 1 0))))
  \end{minted}
 }
\end{frame}

\begin{frame}{\bf Matching Literals}
 Racket literals
 \begin{itemize}
  \item booleans
  \item numbers
  \pause
  \item characters
  \pause
  \item strings
  \item ...
 \end{itemize}

 \pause

 \hl{matching literals $=$ equality test}
\end{frame}

\begin{frame}[plain]
 \begin{center}
  {\LARGE
   Tastes good?
 
   \pause

   \em
   Clean code
  }
 \end{center}
\end{frame}

\begin{frame}[fragile]{{\bf Example 1 --- Debitalization} (2/2)}
 \large
 \begin{table}
  \centering
  \begin{tabular}{ccc}
   bit & $\overset{debitalize}{\longrightarrow}$ & truth \\
   \hline
   0   &                                         & false \\
   1   &                                         & true  \\
  \end{tabular}
 \end{table}

 \pause

 \begin{table}
  \centering
  \begin{tabular}{ccc}
   bit stream & $\overset{debitalize}{\longrightarrow}$ & truth stream   \\
   \hline
   0 1 ...    &                                         & false true ...
  \end{tabular}
 \end{table}
\end{frame}

\section{Matching Data Structures}

\subsection{Matching Built-in DSs}

\begin{frame}[fragile]{{\bf Debitalization} (2a/2)}
 Racketize it

 \pause

 Data
 \begin{itemize}
  \item list of numbers: \mint{scheme}|(list 0 1 ...)|
  \item list of booleans: \mint{scheme}|(list false true ...)|
 \end{itemize}

 \pause

 Code
 {\scriptsize
  \begin{minted}{scheme}
   ;; debits : (listof number) -> (listof boolean)
   ;; debitalizes a bit stream, via isomorphism test
   (define (debits bs)
     (cond ((empty? bs) bs)
           (else (let ((b (first bs))
                       (bs (rest bs)) )
                   (cons (debit b)
                         (debits bs) ) ) ) ) )
  \end{minted}
 }
\end{frame}

\begin{frame}[fragile]{{\bf Debitalization} (2b/2)}
 Bit-wise NOT, AND and OR

 \pause
 
 Racket code
 {\scriptsize
  \begin{minted}{scheme}
   ;; NOTs : (listof number) -> (listof number)
   ;; bit-wise negates a bit stream, via isomorphism test
   ;; ...
  \end{minted}
 }

 \pause

 \mint{scheme}|ANDs| and \mint{scheme}|ORs| ...
\end{frame}

\begin{frame}[fragile]{{\bf Debitalization} (2c/2)}
 ...

 \pause

 {\scriptsize
  \begin{minted}{scheme}
   ;; ANDs : (listof number) (listof number) -> (listof number)
   ;; bit-wise conjoins two bit streams, via isomorphism test
   (define (ANDs bs1 bs2)
     (cond ((or (empty? bs1) (empty? bs2)) empty)
           (else (let ((b1 (first bs1))
                       (bs1 (rest bs1))
                       (b2 (first bs2))
                       (bs2 (rest bs2)) )
                   (cons (AND b1 b2)
                         (ANDs bs1 bs2) ) ) ) ) )

   ;; ORs : (listof number) (listof number) -> (listof number)
   ;; bit-wise disjoins two bit streams, via isomorphism test
   ;; ...

   (debits (NOTs (ORs (ANDs (list 0 1 0))
                      (ANDs (list 1 0 1 0)) ) ) )
  \end{minted}
 }
\end{frame}

\begin{frame}[fragile]{\bf A Second Taste of Pattern Matching}
 Refactor it
 {\scriptsize
  \begin{minted}{scheme}
   ;; debits : (listof number) -> (listof boolean)
   ;; debitalizes a bit stream, via pattern matching
   (define (debits bs)
     (match bs
       ((list) bs)
       ((list b bs ...) (cons (debit b)
                              (debits bs) ) ) ) )
  \end{minted}
 }

 \pause

 {\scriptsize
  \begin{minted}{scheme}
   ;; NOTs : number -> number
   ;; bit-wise negates a bit stream, via pattern matching
   ;; ...
  \end{minted}
 }

 \pause

 \mint{scheme}|AND| and \mint{scheme}|OR| ...
\end{frame}

\begin{frame}[fragile]{\bf Nested Pattern Matching}
 ...

 {\scriptsize
  \begin{minted}{scheme}
   ;; ANDs : (listof number) (listof mumber) -> (listof number)
   ;; bit-wise conjoins two bit streams, via pattern matching
   (define (ANDs bs1 bs2)
     (match bs1
       ((list) bs1)
       ((list b1 bs1 ...)
        (match bs2
          ((list) bs2)
          ((list b2 bs2 ...) (cons (AND b1 b2)
                                   (ANDs bs1 bs2) ) ) ) ) ) )

   ;; OR : (listof number) (listof number) -> (listof number)
   ;; bit-wise disjoins two bit streams, via pattern matching
   ;; ...

   (debits (NOTs (ORs (ANDs (list 0 1 0))
                      (ANDs (list 1 0 1 0)) ) ) )
  \end{minted}
 }
\end{frame}

\begin{frame}{\bf Matching Built-in Data Structures}
 Racket built-in data structures
 \begin{itemize}
  \item lists
  \pause
  \item pairs
  \pause
  \item vectors
  \item ...
 \end{itemize}

 \pause

 \hl{matching (built-in) data structures $=$ isomorphism test}
\end{frame}

\begin{frame}[plain]
 \begin{center}
  {\LARGE
   Tastes good?
   
   \pause

   \em
   Clear code
  }
 \end{center}
\end{frame}

\begin{frame}[fragile]{\bf Motivating Example 2 --- Poker}
 \large
 \begin{table}
  \centering
  \begin{tabular}{c|c}
   \hline
   \multicolumn{2}{c}{card} \\
   \hline
   rank & A, 2, 3, 4, 5, 6, 7, 8, 9, 10, J, Q, K \\
   \hline
   suit & $\spadesuit$, $\heartsuit$, $\diamondsuit$, $\clubsuit$, $\bigstar$ \\
   \hline
  \end{tabular}
 \end{table}
\end{frame}

\subsection{Matching User-defined DSs}

\begin{frame}[fragile]{{\bf Poker} (1a)}
 Racketize it

 \pause

 Data
 \begin{itemize}
  \item structure: \mint{scheme}|struct|
  \pause
  \item string: \mint{scheme}|"A"|, \mint{scheme}|"2"|, \color{strcol}{\verb|"|$\spadesuit$\verb|"|}, ...
 \end{itemize}

 \pause

 Code
 {\footnotesize
  \begin{minted}{scheme}
   (define-struct card (rank suit))
  \end{minted}
 }
\end{frame}

\begin{frame}[fragile]{{\bf Poker} (1b)}\
 {\footnotesize
  \begin{minted}{scheme}
   ;; rank-name : string -> string
   ;; echoes the name of a rank
   ;; ...

   ;; suit-name : string -> string
   ;; echoes the name of a suit
   ;; ...

   ;; card-name : card -> string
   ;; echoes the name of a card
   (define (card-name c)
     (let ((r (card-rank c))
           (s (card-suit c)) )
       (string-append (rank-name r) " of "
                      (suit-name s) "s" ) ) )
  \end{minted}
 }
\end{frame}

\begin{frame}[fragile]{{\bf A Third Taste of Pattern Matching} (1c)}
 Refactor it
 {\footnotesize
  \begin{minted}{scheme}
   ;; card-name : card -> string
   ;; echoes the name of a card
   (define (card-name c)
     (match c
       ((struct card (r s))
        (string-append (rank-name r) " of "
                       (suit-name s) "s") ) ) )
  \end{minted}
 }
\end{frame}

\begin{frame}{\bf Matching User-defined Data Structures}
 Racket user-defined data structures
 \begin{itemize}
  \item via \mint{scheme}|struct|
 \end{itemize}

 \pause

 \hl{matching (user-defined) data structures $=$ isomorphism test}
\end{frame}

\begin{frame}[plain]
 \begin{center}
  {\LARGE
   Tastes good?
   
   \pause

   \em
   Concise code
  }
 \end{center}
\end{frame}

\section{Matching Argument List}

\begin{frame}[fragile]{\bf An Explicit Way to Apply a Function}
 The usual way
 \begin{itemize}
  \item \mint{scheme}|(+ 1 2 3)|
  \item \mint{scheme}|(length (list 1 2 3))|
 \end{itemize}

 \pause

 Say \mint{scheme}|apply| explicitly
 \begin{itemize}
  \item \mint{scheme}|(apply + (list 1 2 3))|
  \pause
  \item \mint{scheme}|(apply length (list (list 1 2 3)))|
 \end{itemize}

 Use it only when necessary ...
\end{frame}

\begin{frame}[fragile]{\bf The Wonder of Variadic Functions}
 \mint{scheme}|(+)| $=$ \mint{scheme}|(apply + (list))| \\
 \pause
 \mint{scheme}|(+ 1)| $=$ \mint{scheme}|(apply + (list 1))| \\
 \pause
 \mint{scheme}|(+ 1 2)| $=$ \mint{scheme}|(apply + (list 1 2))| \\
 \pause
 \mint{scheme}|(+ 1 2 3)| $=$ \mint{scheme}|(apply + (list 1 2 3))| \\
 ... 

 \pause
 Argument list
\end{frame}

\begin{frame}[fragile]{\bf The Magic behind Variadic Functions}\
 {\footnotesize
  \begin{minted}{scheme}
   ;; sum : number ... -> number
   ;; sums an arbitrary number of numbers
   (define (sum . ns)
     (cond ((empty? ns) 0)
           (else (+ (first ns)
                    (apply sum (rest ns)) ) ) ) )
  \end{minted}
 } 

 \pause
 Refactor it?
\end{frame}

\begin{frame}[fragile]{\bf Functions with Optional Arguments}\
 {\footnotesize
  \begin{minted}{scheme}
   ;; greet-person : string string ... -> string
   (define (greet-person g . ps)
     (match ps
       ((list) (string-append g "!"))
       ((list p) (string-append g ", " p "!"))
       ((list p ps ...)
        (string-append g ", " p "! "
                       (apply greet-person (cons g ps)) ) ) ) )
  \end{minted}
 }

 \pause

 An \emph{optional} feature
\end{frame}

\end{document}

